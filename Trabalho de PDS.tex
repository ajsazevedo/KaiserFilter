\documentclass{article}
\usepackage[utf8]{inputenc}
\usepackage{amsmath}
\DeclareUnicodeCharacter{2212}{-}
\DeclareUnicodeCharacter{3C9}{ω}


\title{Trabalho de Processamento Digital de Sinais}
\author{Antonio Azevedo}
\date{June 2019}

\begin{document}

\maketitle

\section{Especificacoes}

3º Projeto: Projete um filtro FIR passa-baixas, de fase linear, pelo método da janela.
Deseja-se uma frequência de corte ($−6$dB) igual a $ωc = π/2$, uma atenuação mínima na banda
rejeitada maior ou igual a $50dB$ e uma região de transição ∆ω < 0,1π. Empregue janela de Kaiser.
Implemente o filtro nas formas direta e em cascata. Represente os coeficientes em ponto flutuante
(ex.: $0,00423578 = 0,423578 × 10−2$) e vá diminuindo o número de casas decimais após a vírgula
nas formas direta e em cascata para verificar a sensibilidade à quantização de parâmetros. Trace a
curva do módulo da resposta em frequência em dB e da fase da resposta em frequência para os
casos de precisão infinita e precisão finita. Em seguida, para a representação em forma direta,
escolha duas das transformações em frequência a seguir (Z−1 = −z−1 ; Z−1 = z-2 ou Z−1 = −z−2 ) e
trace a curva do módulo em dB e da fase da resposta em frequência resultante.

\end{document}
